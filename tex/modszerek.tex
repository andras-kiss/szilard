\chapter{Módszerek}
\pagestyle{headings}

Ebben a fejezetben a teljességre és tömörségre törekedve írom le az alkalmazott módszereket, illetve táblázatosan adom meg a használt anyagok adatait. Abban az esetben, ha a módszer rutinszerű, törekszem rövidített leírásra, és hivatkozom a megfelelő közleményt.

\section{Elektródok}

Munkám során többféle elektródot alkalmaztam. A makroelektródok készítését nem én végeztem, ennek módjáról rövidített leírást adok. A sajátkészítésű mikroelektródok elkészítéséről és jellemzéséről részletesen beszámolok.

\subsection{Antimon mikroelektród}
\paragraph{Az antimon mikroelektródok készítése.}
Egy antimonporral (Szkarabeusz, Pécs, Magyarország) töltött boroszilikát üvegkapillárist (World Precision Instruments, Inc. Sarasota, Florida 34240 USA) gázégőn felhevítettem izzásig, majd vékony üvegkapillárist készítettem, úgy, hogy az üvegkapilláris zárt végét csipesszel megfogva hirtelen mozdulattal vékony szálat húztam belőle. A kapott kb. 0.1 mm átmérőjű antimonnal telt kapillárist mikroszkóp alatt vizsgáltam és kiválasztottam a szakadásmentes részeket, amelyekből 4-5 cm-es darabokkal dolgoztam tovább. Ezeket az alapelektródokat Amepox A és B komponensének 1:1 arányú keverékével (elektromos vezető epoxi) egy elektromos elvezetés szigetelő műanyagától megfosztott rézhuzalhoz rögzítettem, majd szárítószekrénybe (100 \textdegree C) helyeztem a kétkomponensű ragasztó megkötéséig, ami kb. 1 órát vett igénybe.

\begin{figure}[h]
\centering
\includegraphics[width=0.5\textwidth]{img/antimon.png}
\caption{Az 1-es képen az antimon mikroelektród látható, ahol (a) elektromos elvezetés, (b) szigetelő műanyagától megfosztott rézhuzal, (c) elektromos vezető kétkomponensű ragasztó, (d) antimonnal töltött boroszilikát üvegkapilláris, alapelektród. A 2-es képen az alapelektród mikroszkópikus képe látható.}
\label{fig:ionophores}
\end{figure}

\paragraph{Az antimon mikroelektród karakterizálása}\label{karakterizalas}
Az elkészített antimon mikroelektródot a megszokott módszer szerint kalibráltam. A kalibrálást általam készített hét pufferoldat segítségével végeztem, amik körübelül 1 pH egységre álltak egymástól: pH= 3,902; 5,105; 5,442; 6,51; 8,993; 9,464; 10,262. A pontos pH értékeket pH standardokkal kalibrált Methorm pH mérő készülékkel és hozzá tartozó kombinált üvegelektróddal ellenőriztem. 

A cella összeállítása során a mérőelektród a saját készítésű fém mikroelektród, míg referencia elektródként a pH-méréshez használatos kombinált üvegelektród referencia elektródját használtam.
A cella ellenállását feszültségosztó módszerrel határoztam meg. A mérési elrendezést a \ref{fig:feszultsegoszto}. ábra mutatja. 

\begin{figure}[h]
\centering
\includegraphics[width=0.9\textwidth]{img/feszultsegoszto.png}
\caption{Feszültségosztó módszer}
\label{fig:feszultsegoszto}
\end{figure}

\begin{equation}
R_\text{M} = R_\text{in}
\end{equation}
\begin{equation}
V = \frac{E}{2} = \frac{R_\text{in}}{R_\text{M} + R_\text{in}} * E
\end{equation}
\begin{equation} 
\label{eq:A}
R_\text{M} = R_\text{in} (\frac{E}{V} -1)
\end{equation}

A mérést multiméter segítségével végeztem, amit a megfelelő ellenállások használatával és a multiméter által mért feszültség alapján a fenti egyenletbe helyettesítve meghatározható a cella (fém     mikroelektród és az alkalmazott referencia-elektród) ellenállása. Az antimon mikroelektród ellenállását kéziműszer segítségével hajtottam végre. Az antimon-elektródot higanyba mártottam. Eleinte „végtelen ellenállást” mértem, majd az elektródból kis darabot letörve valós értéket tudtam mérni. Feltételezhetjük, hogy az antimon-oxidot nem nedvesíti a higany, evvel magyarázható a mérés kezdeti sikertelensége, míg az antimont (frissen letört elektród esetén) nedvesíti a higany.

Az ellenállás mérés alapján meghatározható az elektród felülete az alábbi egyenlet segítségével:

\begin{equation}
R = \rho * \frac{l}{A}
\end{equation}
ahol $R$ az ellenállás, $\rho$ az anyagi minőségre jellemző fajlagos ellenállás, $l$ a hossz és az $A$ a felület. $R = \rho * \frac{l}{A}$-ből $A$-t kifejezve:
\begin{equation} 
\label{eq:A1}
A = \rho * \frac{l}{R}
\end{equation}
A $\rho$ értéke antimon esetében 417 n\Omega m

A felületet ismerve meg tudjuk határozni az antimon-elektródok átmérőjét:

\begin{equation} 
\label{eq:A2}
A = \Pi \times r^2
\end{equation}\\
Az antimon átmérőjének meghatározása során feltételeztem, hogy az antimon elektród kör keresztmetszetű, és az elektród folytonos illetve egyenlő vastagságú.

\subsection{Szénszál mikroelektród}
Amepox A és B komponensét 1:1 arányban összekevertem, majd egy gyárilag előállított 33 $\upmu$m átmérőjű szénszálat (Specialty Materials Inc. 1449 Middlesex Street Lowell, Massachusetts 01851) a ragasztó segítségével az elektromos elvezetés szigetelő műanyagától elválasztott rézhuzalhoz rögzítettem. A ragasztó megkötéséig szárítószekrénybe helyeztem az elektródokat (kb. 1 óra). Ezután a szénszálat az elektromos elvezetéssel együtt egy húzott végű boroszilikát üvegkapillárisba helyeztem, úgy, hogy a szénszál kb. 5 mm-t lógjon ki a kapillárisból. A kapilláris végének lezárását egy folyékony kétkomponensű epoxi ragasztóval végeztem (helyi barkácsboltból beszerezve), hogy kis mennyiségű folyékony ragasztót a kapilláris végéhez helyeztem, ami a kapilláris hatás miatt a kapillárisba jut, ezáltal egy zárt képezve a kapilláris végén, ami meggátolja a mérendő oldat kapillárisba jutását.
\begin{figure}[h]
\centering
\includegraphics[width=0.5\textwidth]{img/szen33.png}
\caption{(1) Az elektród fotója, ahol (a) elektromos elvezetés, (b) szigetelő műanyagától megfosztott rézhuzal, (c) elektromos vezető epoxi, (d) 33 $\upmu$m átmérőjű szénszál. (2) Az alapelektród mikroszkópikus képe.}
\label{fig:ionophores}
\end{figure}

7$\upmu$m-es szén-mikroelektród készítése:
Egy elektronikai tűfejléc "nőstényébe" (female pin header, Helyi elektronikai boltban vásárolt) egy 7$\upmu$m-es szénszálat (Toray Torayca T700S 24K, 19002 50th Avenue East, Tacoma, WA 98446) beleforrasztottam forrasztóón segítségével úgy, hogy a szénszál kb. 1cm hosszan lógjon ki.A fejléc másik felében egy gyárilag beleforrasztott 1mm átmérőjű rézdrót van, amely az elektromos kontaktust biztosítja.
\begin{figure}[h]
\centering
\includegraphics[width=0.5\textwidth]{img/szenmikro.png}
\caption{(1) Az elektród képe látható, ahol (a) elektromos elvezetés, (b) tüskesor csatlakozóalj (female pin--header), (c) forrasztó--ón , (d) 7 $\upmu$m átmérőjű szénszál. (2) Az alapelektród mikroszkópikus képe.}
\label{fig:ionophores}
\end{figure}

\paragraph{Válaszgyorsaság vizsgálata}

Az oszcilláló reakciók tanulmányozása előtt a mérőműszer válaszgyorsaságát vizsgáltam, hogy biztos legyek benne, hogy elég gyors a kémiai hullámok mérésére. A potenciometriás cella ellenállása és a mérőműszer bemeneti kapacitása együttesen határozza meg az időállandót, mely a cella reagálási sebességét leíró paraméter \cite{kiss2015deconvolution}. A vizsgálatot \emph{,,flip-flop''} típusú 2 V-os 200 ms-os jelgenerátorral végeztem, ami oszcilláló viselkedést mutat. Érdekesség, hogy ez a rendszer ugyanolyan (relaxációs) oszcillátor, mint a vizsgált BZ reakció.

\subsection{Platina mikroelektród}
Gyárilag előállított 25 $\upmu$m átmérőjű platina szálat (Goodfellow Materials) a már jól ismert Amepox A és B komponens 1:1 arányú keverékével ragasztottam egy a szigetelő műanyagától megfosztott elektromos elvezetés réz szálához, amit jól bevált módon kb. 1 órára szárítószekrénybe (100 \textdegree C) helyeztem a kétkomponensű ragasztó megkötéséig.  A megszáradt alapelektródot egy előre kihúzott végű kapillárisba helyeztem, úgy, hogy a platina szál kb. 3-5 mm-re lógjon ki a kihúzott végű kapillárisból. A kapilláris végét folyékony két komponensű ragasztóval zártam le a már fentiekben tárgyalt szénszál mikroelektród készítése alapján. Az elektromos kontaktust ezüst-epoxi (Amepox) biztosította.
\begin{figure}[h]
\centering
\includegraphics[width=0.5\textwidth]{img/platina.png}
\caption{Az 1-es képen az elektród képe látható, ahol (a) elektromos elvezetés, (b) húzott végű boroszilikát üvegkapilláris, (c) elektromos vezető epoxi, (d) folyadékzár  (e) 25 $\upmu$m átmérőjű platinaszál. A 2-es képen az alapelektród mikroszkópikus képe látható.}
\label{fig:ionophores}
\end{figure}


\section{Oszcilláló reakciók tanulmányozása} 

Minden oszcilláló reakciót a következő “hardver” és szoftver felhasználásával hajtottam végre. Indikátor elektród(platina,antimon,szén) Ag/AgCl referencia elektród volt az alkalmazott “hardver”.  Az alkalmazott szoftver az EDAQ Chart (Doig Ave, Denistone East NSW 2112, Australia),  méréstartományként 1V-ot alkalmaztam, mintavételezési frekvenciaként 10 Hz-et állítottam be. A mérési adatokat számítógép rögzítette, ami a mérőcellához volt kapcsolva.

\begin{figure}[h!]
\centering
\includegraphics[width=0.5\textwidth]{img/setup.png}
\caption{A mérési elrendezés. (A) Az Ag/AgCl referencia elektród, (B) a mérőelektród, (C) Belouszov-Zsabotyinyszkij reakció Petri-csészében.}
\label{fig:setup}
\end{figure}


\subsection{Makroelektródokkal}
\begin{enumerate}
\item Vizsgálataim során két különböző oszcilláló reakciót tanulmányoztam az általam készített, az előbbiekben említett elektródokkal. 
Először a következő oszcillátor összetételt vizsgáltam \ref{my-label}.
\begin{table}[h!]
\centering
\caption{A reakció komponenseinek koncentrációi.}
\label{my-label}
\begin{tabular}{llllll}
Komponens                       & Malonsav & Kálium-bromát & Kénsav & Mangán-szulfát & Ioncserélt víz \\
\hline
Koncentráció (M)                & 1        & 0.2           & 5      & 0.125          & -              \\
Térfogat (cm$^3$) & 12       & 9             & 11     & 6              & 1              \\
\end{tabular}
\end{table}\\
Ezek a mennyiségek bemérését meghatározott sorrendben végeztem a következőképpen: Először a kétszer ioncserélt vizet mértem be, majd a malonsavat, kálium-bromátot (Reanal Laborvegyszer Kft., Budapest, Magyarország), kénsavat és végezetül a mangán-szulfátot. A reakció a mangán-szulfát hozzáadása után indult. A reakcióelegyet mágneses keverővel kevertem. Ennek a reakciónak a tanulmázásából messzemenő következtetéseket nem lehet levonni, csak a Field, Noyes és Kőrös által publikált eredmények reprodukálása \cite{noyes1972oscillations} volt a cél, amint azt már a \ref{bromatoszcillator} című fejezetben említettem.
\item A kevertetett reakció esetén spektrofotometriásan meghatároztam a piros illetve a kék szín abszorbciós maximumát.
\end{enumerate}

\subsection{Mikroelektródokkal} \label{mikroelektrod}

A Field, Noyes és Kőrös által elért eredmények reprodukálása utána \cite{noyes1972oscillations} keveretlen közegben vizsgáltam a reakciót.
A továbbiakban vizsgált oszcilláló reakciót kevertetés nélkül, a következő összetételű oldatokkal dolgoztam: 

\begin{itemize} \label{komponensek}
\item A oldat: 67ml H$_2$O + 5g KBrO$_3$ + 2ml H$_2$SO$_4$
\item B oldat: 1g malonsav + 10ml H$_2$O
\item C oldat: 1g KBr + 10ml H$_2$O
\end{itemize}

Az antimon mikroelektród esetében erősítőt kellett beépíteni az áramkörbe, a terhelési hiba kiküszöbölése végett. A műveleti erősítővel az áramkör kapcsolási rajza az \ref{fig:erosito} ábrán látható. Az antimon jelének felerősítése egy TL082-műveleti erősítővel valósult meg a gyakorlatban.
\begin{figure}[h!]
\centering
\includegraphics[width=0.8\textwidth]{img/erosito2.png}
\caption{Feszültségosztó módszer}
\label{fig:erosito}
\end{figure}


\subsection{Pásztázó elektrokémiai mikroszkóp alkalmazása}

\begin{figure}[h]
\centering
\includegraphics[width=0.8\textwidth]{img/secm.png}
\caption{(A) Léptetőmotorok, (B) indikátor elektród (C) referencia elektród, (D) kamera, (E) nagy bemeneti impedanciájú feszültségmérő.}
\label{fig:secm}
\end{figure}

A mikroelektródok sikeres elkészítése, valamint alkalmazása után az elektrokémiai jelhez képet rendelhetünk, ezáltal megkapva az adott pont elektrokémiai képét, amit következőképpen csináltam. A mérésről videót (8MP Sony Exmor IMX179 1/3,2 inch CMOS szenzor, LG Nexus 5 típusú mobiltelefon) készítettem, amit a pásztázási vonaltól 10 cm-re helyeztem el és 45$^0$-os szöget zárt be azzal. A mikroszkóp által a potenciometriás mikroelektród 500 $\upmu$m/s sebességgel haladt a felületen, 100 $\upmu$m lépésközzel, 10 mm pásztázási úton. A mérés befejeztével a videót a ,,Blender'' \cite{blender} nevezetű program segítségével képkockákra bontottam. A mikroelektród hegyénél lévő 1 pixel $\times$ 1 pixel-es képkockát a teljes mérés során kivettem. A kivett képkockák száma megegyszezik a videó hosszával másodpercben véve.

\begin{equation} 
\textrm{Képkockák száma = Videó hossza(s) * Képkocka/másodperc (fps)}
\end{equation}

A telefon által rögzített videók 29.93 fps-al történt. Elegendő másodpercenként egy képkockát kivenni a videóból, ez a gyakorlatban minden 30. képkocka kivételével valósult meg.
Az általam készített elektrokémiai képhez kb. 550 képkockát használtam fel, majd ezeket egymás mellé helyezve az ,,Imagemagic'' \cite{imagemagick} nevű program segítségével megkapjuk az elektrokémiai képet, ami jól látható a ferroin katalizált BZ-reakció esetében.  
