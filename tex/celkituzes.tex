\chapter{Célkitűzés}
\pagestyle{headings}

Szakdolgozatom fő célja a folyadékfázis felületén történő potenciometriás térképezés lehetőségének feltárása a keveretlen Belouszov-Zsabotyinszkij (BZ) oszcilláló reakció redox-potenciál térképezésének példáján keresztül. Célom elérése érdekében először megismerkedtem a potenciometriás méréstechnikával, majd a korábbi, a témában mérföldkőnek számító elektrokémiai vizsgálatokat ismételtem meg. Először makroméretű elektródokkal kíséreltem meg a kevert BZ reakció vizsgálatát. Majd ugyanezen kísérleteket elvégeztem a makroelektródok mikroméretű változataival. A mikroelektródokat az eredmények kiértékeléséhez jellemeznem kellett. Végezetül dolgozatom fő célját valósítottam meg egy, a BZ reakcióról készült redoxpotenciál kép megalkotásával. Ebben a kísérletben egy sajátkészítésű, újszerű, szigeteletlen szénszál mikroelektródot alkalmaztam PEKM mérőcsúcsként. Pontokba foglalva az alábbi célokat tűztem ki magam elé a munka megkezdése előtt és során:

\begin{enumerate}
\item Kevert BZ reakció vizsgálata makroméretű potenciometriás elektródokkal.
\item Az ennek megfelelő potenciometriás mikroelektródok elkészítése és jellemzése.
\item A készített mikroelektródok alkalmazhatóságának ellenőrzése a kevert reakció tanulmányozásával.
\item Pontszerű mérés ugyanezen mikroelektródokkal a keveretlen BZ reakcióban. Az elektrokémiai és az optikai módszer összevetése.
\item Pásztázó elektrokémiai mikroszkóp alkalmazása: az első potenciometriás tér-idő kép a BZ reakcióról, ennek összehasonlítása az optikai tér-idő képpel.
\end{enumerate}

Az első négy pont korábbi munkák reprodukálása, a negyedik pont jelen munka újdonsága. 

