\chapter{Célkitűzés}
\pagestyle{headings}
 Szakdolgozatom fő célja a folyadékfázis felületén történő potenciometriás térképezés lehetőségének feltárása a keveretlen Belouszov-Zsabotyinkszkij (BZ) oszcilláló reakció redox-potenciál térképezésének példáján keresztül. Célom elérése érdekében először megismerkedtem a potenciometriás méréstechnikával, majd a korábbi, mérföldkőnek számító elektrokémiai vizsgálatokat ismételtem meg. Először makroméretű elektródokkal kíséreltem meg a kevert BZ reakció vizsgálatát. Majd ugyanezen kísérleteket elvégeztem a makroelektródok mikroméretű változataival. A mikroelektródokat az eredmények kiértékeléséhez jellemeznem kellett. Végezetül dolgozatom fő célját valósítottam meg egy, a BZ reakcióról készült redoxpotenciál kép megalkotásával. Ebben a kísérletben egy sajátkészítésű, újszerű, szigeteletlen szénszál mikroelektródot alkalmaztam PEKM mérőcsúcsként. Pontokba foglalva az alábbi célokat tűztem ki magam elé:
 \begin{itemize}
 \item BZ reakció vizsgálata makroméretű potenciometriás elektródokkal.
 \item Az ennek megfelelő potenciometriás mikroelektródok elkészítése és jellemzése.
 \item A készített mikroelektródok alkalmazhatóságának ellenőrzése a keveretlen reakció tanulmányozásával.
 \item Pontszerű mérés ugyanezen elektródokkal a keveretlen BZ reakcióban.
 \item Pásztázó elektrokémiai mikroszkóp alkalmazása: az első potenciometriás tér-idő kép a BZ reakcióról.
 \end{itemize}
 