\chapter{Eredmények}
\pagestyle{headings}

\def\s{0.5}
\begin{figure}
%% trim = top left bottom right
\centering
\includegraphics[trim = 15mm 60mm 0mm 30mm, clip, width=\s\textwidth, angle=-90]{img/spacetime.eps}
\caption{(A) Electrochemical space-time plot overlayed on top of the corresponding optical one.
The potential of the carbon fiber microelectrode is shown as a function of spatiotemporal coordinates.
Redox indicator was ferroin.
The optical spatiotemporal image was grayscaled to allow better visibility when overlaying the electrochemical data.
The brighter stripes correspond to the oxidizing waves.
(B) The second scan cycle.
Blue line: forward scan, red line: backwards scan.
%Note the shape of the peaks on the red curve, which is measured in the direction that is opposite to the chemical wave travel.
%The elongated portion of these peaks where the potential is decreasing, is not due to 
}
\label{fig:spatiotemporal}
\end{figure}
