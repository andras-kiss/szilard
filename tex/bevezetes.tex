\chapter{Bevezetés}
\pagestyle{headings}
Az 1980-as években dolgozta ki Rohrer és Binning az első pásztázó mikroszkópiás módszert, a pásztázó alagút mikroszkópiát (PAM, Scanning Tunneling Microscopy STM). Munkájuk kiemelkedő fontosságát jelzi, hogy néhány évvel később nekik itélték a fizikai Nobel-díjat. Az új mikroszkópiás módszer kifejlesztése után viszonylag gyorsan jelent meg a többi, hasonló elven működő felületvizsgáló technika. Elsőként az atomerő  mikroszkópia (AEM, Atomic Force Microscopy AFM), később az elektrokémikusok által kifejlesztett pásztázó elektrokémiai mikroszkópia (PEKM, Scanning ElectroChemical Microscopy SECM). Az elekrokémiai mikroszkóp kifejlesztése Allen J. Bard és munkatársai nevéhez fűzhető. A pásztázó mikroszkópiás módszerek alapja a pásztázó mérőcsúcs, mely egy bizonyos előre meghatározott program szerint végighaladva a vizsgált rendszeren térbeli képet készít a mérőcsúcs által mért paraméterről. A PEKM mérőcsúcs általában egy ultramikro elektród, melynek amperometriás vagy potenciometriás jele egy adott anyagféleség lokális, a mérőcsúcsnál lévő koncentrációjától függ.

Az amperometriás detektáláson alapuló közlemények száma jóval nagyobb a potenciometriás mérőcsúcsot használó közleményekénél. Ez az amperometriás mikroelektródok több előnyös tulajdonságára vezethető vissza. A mérőcsúcsként használt ultramikro méretű munkaelektródok készítése sokkal könnyebb, és robusztusabb mérőcsúcsot eredményez. Könnyebb a Z-irányú pozícionálás, a visszacsatolás jelenségének köszönhetően. A potenciometriás ion-szelektív mikroelektródok sokkal törékenyebbek, nehezebben használhatóak. Számos olyan jelenség akad azonban, amit más módszerrel nem lehet megfelelően vizsgálni, ilyenek például az \emph{in situ} PEKM korróziós vizsgálatok, melyek során általában egy konkrét ionféleség térbeli aktivitás eloszlására vagyunk kíváncsiak egy korrodálódó minta felett. Egy másik jó példa a hidrogén-ion aktivitás térképezése olyan folyamatokban, melyekben a pH erősen függ a térkoordinátáktól, például sejtek vagy sejttelepek közvetlen környezete a tápoldatban. Éppen ezért érdemes a potenciometriás PEKM fejlesztésével foglalkozni.

2017-ben kapcsolódtam be a Pécsi Tudományegyetem Általános és Fizikai Kémia Tanszékén folyó ez irányú kutatásokba. A potenciometriás PEKM egy új lehetséges alkalmazási területének hatékonyságát vizsgáltam; munkámban folyadékfázis felületének potenciometriás térképezését kíséreltem meg. A PEKM technikát az esetek túlnyomó többségében elektrolitban lévő szilárd felületek vizsgálatára alkalmazzák. A folyadékfázis felületének PEKM térképezése egy eddig még nem vizsgált, új alkalmazási terület, mely több érdekes folyamat megfigyelését tenné lehetővé. Példaként a jól ismert Belouszov-Zsabotyinszkij (BZ) oszcilláló reakciót használtam, melyben erősen tér- és időkoordináta függő például a bromid-ion aktivitás és a katalizátor oxidációs állapotától függő redoxpotenciál. Tér- és időkoordináta függvényében mértem a redoxpotenciált a reakcióelegy felületén. A kvázi kétdimenziós keveretlen BZ-reakció elegyben vertikális koncentráció változással nem kell számolni, ezért Z-irányú térképezésre nem volt szükség. Továbbá ugyanezen okból kifolyólag elegendő a legfelső, kevesebb mint 1 mikrométer vastagságú oldatréteg térképezése. További kisérleti előnyt jelent a folyadékfázis felületének tökéletes simasága. Ezenkívül, mivel a mikroelektród csak kevesebb mint 1 mikrométeres mélységbe hatol az oldatba, szigetelésre nem volt szükség. Ez jelentősen kisebb méretű mikroelektród készítését tette lehetővé, mely igencsak fontos a keveretlen oszcilláló reakció vizsgálata során, hiszen minél kisebb a mikroelektród, annál kisebb konvektív hatása van mozgás közben. Dolgozatomban felderítem a folyadékfázis felületén való potenciometriás pásztázás lehetőségét. Példaként bemutatom a technikával nyert első potenciometriás tér-idő képet a BZ-reakcióról és összehasonlítom a már jól ismert opikai tér-idő képpel.
