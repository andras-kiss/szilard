\chapter{Összefoglalás és következtetések}
\pagestyle{headings}
A potenciometriát a Belouszov-Zsabotyinszkij reakció felfedezése óta használják annak tanulmányozására. Térbeli információt azonban nem szolgáltatott a módszer. Ellenben például a 2D fotometriával vagy a tomográfiával, ahol viszont kémiai információt nem kapnak az optikai kép mellé. Szakdolgozatomban az optikai és elektrokémiai módszerek előnyeinek kombinálását mutattam be. A potenciometriás PEKM technikát egy kisméretű mikroelektróddal alkalmaztam egy kvázi két-dimenziós BZ-reakcióban. Egy kamera által rögzített optikai képhez a mérőműszer által mért elektrokémiai információt rendelve meggyőződtem róla, hogy valóban a BZ-reakció felületén kialakuló redox--potenciál képhez jutottam. A \ref{celkituzes}. fejezetben említetteket sikeresen megvalósítottam.

A munkám során nem a reakció kinetikáját vizsgáltam, hanem főleg a PEKM technika egy új alkalmazását szerettem volna bemutatni; az oldatfázis felületén való pásztázást potencimetriás mikroelektróddal. A PEKM technika előnye, hogy a mérőcsúcsot kicserélve új, szelektív kémiai információt hordozó térbeli felbontású kép készíthető a vizsgált reakcióról. Ezt kihasználva a jövőben például a bromid-ion aktivitás térképezhető a dolgozatomban bemutatott technikával bromid-ion szelektív elektród alkalmazásával.

Dolgozatom témája egy jó példa az interdiszciplinaritásra. Elektrokémiai módszerek újszerű alkalmazásával és továbbfejlesztésével esetleg új jelenségek figyelhetőek meg más tudományágakban, például biológiában. A technika alkalmas lehet embrió felületi vizsgálatára, aminek segítségével közelebb kerülhetünk a biológiai mintázatképződés nagy kérdésének megválaszolásában; Hogy alakulhat ki az élő szervezetek nagyfokú szervezettsége olyan viszonylag egyszerű kiindulásból, mint egy zigóta? Ehhez hasonló kísérletek talán érdekes, új információt szolgáltathatnak a kérdés megválaszolásához. A PEKM új alkalmazása fontos első lépése lehet egy nagyobb kísérlet sorozatnak.
