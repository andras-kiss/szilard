\chapter{Összefoglalás és következtetések}
\pagestyle{headings}
A potenciometriát a Belouszov-Zsabotyinszkij reakció felfedezése óta annak tanulmányozására használták, azonban térbeli információt nem szolgáltatott a módszer. Ellenben például a 2D fotometriával vagy a tomográfiával szemben, ahol azonban elektrokémiai információt nem kapnak az optikai kép mellé. Szakdolgozatomban az optikai és elektrokémiai képek kombinálását mutattam be, a potenciómetriás PEKM technikát egy  kisméretű mikroelektróddal alkalmaztam  egy kvázi két-dimenziós BZ-reakcióban. A kamera által rögzített optikai képhez a mérőműszer által mért elektrokémiai információt rendelve a BZ-reakció felületén kialakuló elektrokémiai képhez jutottam.
A \ref{celkituzes}. fejezetben említetteket sikeresen megvalósítottam.